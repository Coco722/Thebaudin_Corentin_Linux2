\documentclass[a4paper,12pt]{article}
\usepackage[utf8]{inputenc}
\usepackage[francais]{babel}
\usepackage[T1]{fontenc}
\usepackage[pdftex]{graphicx}
\usepackage{url}


\setlength{\parindent}{0cm}
\setlength{\parskip}{1ex plus 0.5ex minus 0.2ex}
\newcommand{\hsp}{\hspace{20pt}}
\newcommand{\HRule}{\rule{\linewidth}{0.5mm}}
%opening


\begin{document}

\begin{titlepage}
  \begin{sffamily}
  \begin{center}

    % Upper part of the page. The '~' is needed because \\
    % only works if a paragraph has started.
    % \includegraphics[scale=1]{univangers.jpg}~\\[1.5cm]

    \textsc{\LARGE Université d'Angers}\\[2cm]

   

    % Title
    \HRule \\[0.4cm]
    { \huge \bfseries TP Linux 2}
    \HRule \\[2cm]
    

    % Author and supervisor
    \begin{minipage}{0.4\textwidth}
      \begin{flushleft} \large
        THEBAUDIN \textsc{Corentin}\\
      \end{flushleft}
    \end{minipage}
    

    \vfill
    \HRule\\[2cm]
    % Bottom of the page
    {\large 16 novembre 2016}

  \end{center}
  \end{sffamily}
\end{titlepage}
\clearpage

\tableofcontents

\clearpage



\section{TP1}
\paragraph{Hyperviseur Virtual Box}

Installation de la virtual box via la commande:

apt-get install virtualbox-5.0

\paragraph{Machine virtuelle}

Téléchargement de l'image du disque via la commande:

wget http://cdimage.debian.org/debian-cd/8.6.0/multi-arch/iso-cd/debian-8.6.0-amd64-i386-netinst.iso

L'invité est démarré correctement, les logiciels ont bien été installés via la commande suivante:

aptitude install lynx sudo tcpdump vim

\paragraph{Sudo}

La commande visudo vérifier la syntaxe du fichier avant d'écraser l'ancien. Au contraire, la commande vim /etc/sudoers ne vérifie pas les changements du fichier modifié, on pourrait ainsi bloquer la fonctionnalité sudo et donc ne plus avoir accès au fichier sudoers. L'accès serait donc bloqué.
La commande sudo fonctionne bien après avoir intégré l'utilisateur dans le groupe et décommenté la ligne de mise en route du sudo.

\paragraph{Clonage}

Les clones sont bien réalisés et fonctionnels.

\paragraph{Snapshot}

- Réalisation d'un snapshot de la VM, puis utilisation de la commande en root:

rm -rf / --no-preserve-root

- Après une tentative de redémarrage de la VM, celle-ci ne peut se relancer.

- Restauration de la VM au moment de la snapshot (restaurer l'instantané)

- Relancement de la machine possible

\paragraph{Configuration réseaux}

- Modification de la configuration de la VM: changement du type ``NAT'' en type ``pont'' 
- Réalisation des commandes ``ip addr'' et ``lynx''

Dans lynx, l'accès à l'addresse n'est pas possible. Cela est du aux restrictions d'accès au site.



\clearpage

\end{document}